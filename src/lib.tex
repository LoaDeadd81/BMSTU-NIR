\addcontentsline{toc}{chapter}{СПИСОК ИСПОЛЬЗОВАННЫХ ИСТОЧНИКОВ}

\renewcommand\bibname{СПИСОК ИСПОЛЬЗОВАННЫХ ИСТОЧНИКОВ}
\begin{thebibliography}{}

\bibitem{ML_no_wors} Бурков А. Машинное обучение без лишних слов / А. Бурков. --- СПб.: Питер, 2020. --- 192 с.

\bibitem{ML_for_rank} Городилов А. Методы машинного обучения для задачи ранжирования / А. Городилов // Труды Третьей Российской конференции молодых ученых по информационному поиску. – 2009. – С. 12--20.

\bibitem{ML_for_SE} Борисовская А. А. Применение методов машинного обучения в задачах ранжирования в поисковых информационных системах. / А. А. Борисовская // Информационные технологии в управлении. – 2021. – С. 1--13.

\bibitem{LR} Cossock D. Subset ranking using regression / Cossock D., Zhang T.// In Proceedings of the 19th Annual Conference on Learning Theory (COLT). --- 2006. --- P. 605–619.

\bibitem{RankSVM} Herbrich R. Support vector learning for ordinal regression / Herbrich R., Graepel T., Obermayer K. // In International Conference on Artificial Neural Networks. --- 1999. ---
P. 97--102.

\bibitem{LamdaRank} Burges C. J. Learning to rank with nonsmooth cost functions / Burges C. J., Ragno R., Le Q. V. // Advances in Neural Information Processing Systems 19. --- 2007. --- P. 193--200.

\bibitem{ListNet} Z. Cao Learning to rank: From pairwise approach to listwise approach / Z. Cao, T. Qin,
T.-Y. Liu et al. // Proceedings of the 24th International Conference on Machine Learning. --- 2007. --- P. 129--136.

\bibitem{metrics} Беляков А.В Метрики качества в задачах ранжирования информации / Беляков А.В // Информационные технологии, межвузовский сборник научных трудов. --- Рязань: Рязанский государственный радиотехнический университет имени В.Ф. Уткина. --- 2019. --- C. 36--39.

\bibitem{cmp} Niek Tax A Cross-Benchmark Comparison of 87 Learning to Rank Methods / Niek Tax, Sander Bockting, Djoerd Hiemstra // Information Processing and Management. ---2015. P. 757--772.

\end{thebibliography}


