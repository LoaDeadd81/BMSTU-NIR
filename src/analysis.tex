\chapter{Анализ предметной области}

\section{Основные понятия}
Машинное обучение --- раздел информатики, посвященный созданию алгоритмов, опирающихся на набор данных о каком-либо явлении. Данные получают из естественной среды, создают вручную или генерируют другим алгоритмом. Методы машинного обучения формируют статистическую модель на основе специально подобранных обучающих данных, которую потом используют для решения практических задач~\cite{ML_no_wors}.


Выделяется три основных способа обучения: с учителем, без учителя и с подкреплением. Способы и их применения проиллюстрированы на рисунке~\ref{img:ml-learn}.

\imgScale{0.45}{ml-learn}{Способы обучения}

В обучении с учителем набор данных организован как коллекция размеченных образцов вида <<известный вход --- известный выход>>, называемый обучающей выборкой.  Входные данные являются векторами признаков, описывающих некоторые характеристики образца. Выходом же является предсказание модели, которое может быть элементом конечного множества, вещественным числом или более сложной структурой, например, значение метки, если используется регрессия. Существует некоторая зависимость между входными и выходными данными, которую в результате обучения должна отражать модель. После обучения она способная выдать достаточно точный результат для любых входных данных. Данный подход в основном используется для решения задач регрессии и классификации~\cite{ML_no_wors}.

Обучение без учителя --- класс задач машинного обучения, в которых известны только описания множества объектов, подаваемых на вход модели, и требуется обнаружить внутренние взаимосвязи между этими объектами. При обучении на вход модели подаётся вектор признаков, а целью алгоритма обучения без учителя является --- создание модели, которая принимает входные параметры и преобразует их в значение, которое можно использовать для решения практической задачи. Данный подход используются для задач кластеризации, уменьшения размерности и выявления аномалий~\cite{ML_no_wors}.

Обучение с подкреплением --- это раздел машинного обучения, в котором предполагается, что машина «живет» в каком-то виртуальном окружении и способна воспринимать его как вектор характеристик. Вектор содержит данных о <<мире>> необходимые для обучения модели. Компьютер взаимодействует с этим окружением, параллельно обучаясь, и получает вознаграждение или штраф за выполнение действий, которые помогают правильно корректировать веса модели, то есть обучается методом проб и ошибок. Цель алгоритма обучения с подкреплением --- выучить линию поведения, которая приводит к максимальному ожидаемому вознаграждению~\cite{ML_no_wors}.

\section{Применение машинного обучения в поисковых системах}

Работа поисковой системы заключается в том, чтобы по запросу пользователя найти документы, наилучшим образом подходящие его запросу. При этом основными критериями качества результатов информационного поиска являются полнота, точность и оперативность. Сайты в поисковой выдаче сортируются в соответствии с факторами, используемыми поисковой машиной, то есть проходят процедуру ранжирования~\cite{ML_for_rank}.

Существует множество методов подбора формулы для ранжирования, но одним из самых популярных --- на основе машинного обучения, а именно~---~обучение ранжированию с учителем. Целью этих методов является подбор ранжирующей модели по обучающей выборке, которая способна наилучшим образом приблизить и обобщить способ ранжирования на новые данные. Для получения набора примеров используются асессоры, которые оценивают степень релевантности документа запросу. Данные для обучения характеризуются большим объёмом (100 -- 500 тысяч примеров) и большой размерностью (100 -- 1000 признаков)~\cite{ML_for_rank}.

