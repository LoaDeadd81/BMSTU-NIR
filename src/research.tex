\chapter{Исследование существующих алгоритмов \\обучения ранжированию}

\section{Классификация алгоритмов}

Существующие алгоритмы обучения ранжированию делятся на три группы по подходу к обучению~\cite{ML_for_SE}: поточечный, попарный и списочный.

Поточечный подход --- частный случай задачи регрессии или классификации, если множество оценок конечно, в ходе ее решения возможно использовать метод градиентного спуска, опорных векторов и т.~д. На практике этот подход дает не очень качественный результат, так как каждый документ ранжируется независимо от других. В качестве примеров для обучения используют пары <<признаки~---~значение релевантности>>. Данная задача может быть аппроксимирована задачей регрессии --- учитывая единственную пару <<запрос~---~документ>>, спрогнозируйте ее оценку. Точечный подход направлен на изучение функции, предсказывающей реальное значение или порядковый номер документа, используя функцию потерь.

При попарном подходе обучение ранжированию сводится к построению бинарного классификатора, которому на вход поступают два документа, соответствующих одному и тому же запросу, и требуется определить, какой из релевантнее. Классификатор должен принимать два документа в качестве входных данных, и цель состоит в том, чтобы минимизировать функцию потерь. Недостатком также является то, что данный подход не учитывает все документы запроса. Помимо этого целью обучения таких алгоритмов является минимизация ошибок в классификации пар документов, а не минимизация ошибок в ранжировании документов, процесс обучения требует больших вычислительных затрат, количество сгенерированных пар документов в значительной степени варьируется от запроса к запросу, что приводит к обучению модели, ориентированной на запросы с большим количеством пар.

Списочный подход заключается в построении модели, на вход которой поступают сразу все документы, соответствующие запросу, а на выходе получается их перестановка. Подгонка параметров модели осуществляется для прямой максимизации одной метрик ранжирования. Но это часто затруднительно, так как метрики ранжирования обычно не непрерывны и недифференцируемы относительно параметров ранжирующей модели, поэтому прибегают к максимизации неких их приближений или нижних оценок. С точки зрения поставленной задачи ранжирования, списочные методы являются методами, которые решают исходную постановку задачи. Является одним из самых лучших методов на данный момент, но они сложны в разработке и подборе обучающих наборов данных

\section{Алгоритмы обучения ранжированию}

В данном разделе будут рассмотрены несколько популярных алгоритмов из каждой категории: Linear Regression (поточечный), Ranking SVM (попарный), LambdaRank(попарный), ListNet(списочный). 

Перед описанием алгоритмов стоит введём следующие обозначения:
\begin{itemize}[label=---]
	\item $q_{i}$ --- $i$-ый поисковый запрос;
	\item $D_{i}$ --- список всех документов, которые ассоциированы с $q_{i}$ запросом;
	\item $d_{i,j}$ --- $j$-ый документ из списка $D_{i}$;
	\item $\pi_{i}$ --- отсортированный список $D_{i}$;
	\item $y_{i,j}$ --- метка, показывающая на сколько документ $d_{i,j}$ релевантен к запросу $q_{i}$;
	\item $y_{i}$ --- вектор меток $y_{i,j}$.
\end{itemize}

На этапе ранжирования методы имеют схожий алгоритм. Для каждого документа $d_{i,j}$ вычисляется рейтинг релевантности, который зависит от вектора признаков документа $x_{i,j}$ и параметров метода ранжирования $\omega$. Затем рейтинги сортируются по убыванию и получается отранжированный список документов. Признаки документов зависят не только от описания документов, но и от функции от поискового запроса и
документа.

В ранжирования признаки документов формируются основываясь не только на документах или только на запросах, но и как функции от поискового запроса и документа. Такие признаки способны информативно отражать степень релевантности документа запросу. Зачастую используются такие признака:
\begin{itemize}[label=---]
	\item текстовые --- описывают количество и место слов встречающихся в документе и запросе. Используются меры, такие как TF-IDF, используемая для оценки важности слов в контексте документа, BM 25 и различные языковые модели;
	\item ссылочные --- описывают количество ссылок на документ и сколько внутри самого документа полезных ссылок. Мера PageRank назначает каждому документу некоторое численное значение, измеряющее его «важность» или «авторитетность» среди остальных документов основываясь на информации о гиперссылках.  HITS позволяет находить страницы, соответствующие запросу пользователя, на основе информации, заложенной в ссылки. Идея основана на предположении, что гиперссылки кодируют значительное количество скрытых сайтов соответствующий запросу пользователя;
	\item кликовые --- описывают количество кликов на документ в общем и по определенному запросу. 
\end{itemize}

\subsection{Linear Regression}

Рассматриваем метод обучения на основе регрессии для решения задачи оптимизации метрик DCG~\cite{LR}. Пусть $F$ --- функциональное пространство, содержащее функции $X \times S \to F$, где $X$ --- множество векторов представления документов, $S$ --- множества всех конечных подмножеств $X$. Произвольно созданы множества $S_{1}, \dots, S{n}$, где $S_{i} = \{x_{i,1}, \dots, x_{i,m}\}$, с соответствующими оценками $\{y_{i,j}\}_{j} = \{y_{i,1}, \dots, y_{i,m}\}$. Тогда для решения проблемы ранжирования можно использовать простой подход, основанный на регрессии:
\begin{equation}
	\label{eq:LR1}
	\hat{f}=\arg \min _{f \in \mathcal{F}} \frac{1}{n} \sum_{i=1}^n[\sum_{j=1}^m(f(x_{i, j}, S_i)-y_{i, j})^2].
\end{equation}

Регрессия --- задача прогнозирования метки по набору признаков объектов в общем случае. Метка здесь может принимать любое значение. Алгоритм моделируют зависимость меток от признаков, чтобы определить закономерности изменения меток в зависимости от характеристик объекта. При обучении алгоритму на вход подаются примеры с известными метками. Результатом работы является функция способная прогнозировать значение метки, для новых данных. 

Для возможности предсказания меток в регрессии гиперплоскость проводиться так, чтобы оказаться как можно ближе к обучающим примерам~\cite{ML_no_wors}. Для это требуется минимизировать целевую функцию~(\ref{eq:LR1}). Выражение \\$(f(x_{i, j}, S_i)-y_{i, j})^2$ называют функцией потерь, в данном случает используется квадратичная функция потерь. Она определяет величину штрафа за неправильную классификацию. В этом алгоритме функция стоимости определена как средняя потеря. Минимизировав целевую функцию получим оптимальные параметры модели, для предсказания меток.

Однако этот метод прямой регрессии не подходит для крупномасштабных задач ранжирования, таких как веб--поиск, для которых требуется ранжировать множество элементов, но важны только страницы с самым высоким рейтингом. Это связано с тем, что метод уделяет равное внимание релевантным и нерелевантным страницам. На самом деле, следует уделять больше внимания страницам с самым высоким рейтингом. Оценки страниц с низкой релевантностью не нуждаются в большой точности, до тех пор, пока мы не переоцениваем их, чтобы эти страницы отображались на верхних позициях в поиске. С учётом этик рассуждений стоит изменить формулу~(\ref{eq:LR1}) следующим образом:
{\begin{equation}
	\label{eq:LR2}
	\hat{f}=\arg \min _{f \in \mathcal{F}} \frac{1}{n} \sum_{i=1}^n L(f, S_i,\{y_{i, j}\}_j),
\end{equation}}
где для $S = \{x_{1}, \dots, x_{m}\}$, с соответствующим $\{y_{j}\}_{j}$, имеем целевую функцию:
\[
\begin{aligned}
	& L(f, S,\{y_j\}_j) \\
	= & \sum_{j=1}^m w(x_j, S)(f(x_j, S)-y_j)^2+u \sup _j w^{\prime}(x_j, S)(f(x_j, S)-\delta(x_j, S))_{+}^2,
\end{aligned}
\]
где $u$ --- неотрицательный параметр. Весовая функция $w(x_j, S)$ выбрана таким образом, чтобы она фокусировалась только на наиболее важных документах. Во второй части уравнения мы выбирается $w'(x_j, S)$ так, чтобы она фокусировалась на страницах, не охваченных $w(x_j, S)$. $\delta(x_j, S)$ выбирается в качестве нижнего порога. Важной особенностью является то, что, хотя $m$ часто очень велико, количество точек, при которых $w(x_j, S)$ отлично от нуля, зачастую достаточно мало. Более того, $(f(x_j, S)-\delta(x_j, S))_{+}$ не равно нулю только тогда, когда $f(x_j, S) \geq \delta(x_j, S)$.Следовательно, полностью игнорируются низкоранговые страницы,что делает алгоритм вычислительно эффективным даже при большом $m$. Формулы~(\ref{eq:LR1}) и~(\ref{eq:LR2}) \cite{LR}. 

\subsection{Ranking SVM}

Ключевая идея алгоритма, предложенного Р. Хербрихом~\cite{RankSVM}, заключается в использовании метода SVM для попарного сравнения документов на то, какой из элементов более релевантный. 

Метод опорных векторов --- набор алгоритмов обучения с учителем, используемый для задач классификации и регрессии. Суть метода заключается в построении гиперплоскости, чтобы расстояние от неё до ближайшей точки было максимальным~\cite{ML_no_wors}. Это эквивалентно тому, что сумма расстояний до гиперплоскости от двух ближайших к ней точек, лежащих по разные стороны от неё, максимальна. На основе взаимопритяжения точки, представляющей объект, и гиперплоскости происходит  классификация.

При обучении на вход подаются набор пар вида <<точка --- метка класса>>, где метка класса это 1 или -1. Строится гиперплоскость имеющая вид $w\cdot x - b = 0$, где $x$ ---  входной вектор признаков. Так как ищется оптимальное разделение, особый интерес представляют опорные вектора и гиперплоскости, параллельные оптимальной и ближайшие к опорным векторам двух классов. 
Можно показать, что эти параллельные гиперплоскости могут быть описаны следующими уравнениями:
\begin{equation}
		\label{eq:RSVM1}
	\begin{aligned}
		w \cdot x - b = 1  \\
		w \cdot x - b = - 1 .
	\end{aligned}
\end{equation}

Ширину полосы легко найти из соображение геометрии. Она равна $\frac{2}{||w||}$. получается что задача состоит в минимизации $||w||$, так как при чем меньше  $||w||$ , тем больше расстояние между этими двумя гиперплоскостями.Большой зазор способствует лучшему обобщению, то есть тому, насколько хорошо модель
будет классифицировать новые данные. Чтобы исключить все точки $x_i$ из полосы необходимо включить следующие ограничения:
\begin{equation}
	\label{eq:RSVM2}
	\begin{aligned}
	w \cdot x - b \geq 1, \text{если $y_i = +1$}  \\
	w \cdot x - b \leq -1, \text{если $y_i = -1$} ,
	\end{aligned}
\end{equation}
где $y_i$ --- метка.
Уравнения (\ref{eq:RSVM2}) можно записать кратко в виде:
\begin{equation}
	\label{eq:RSVM3}
	y_i(w \cdot x - b) \geq 1.
\end{equation}

В случает линейной неразрешимости, алгоритму позволяется допускать ошибки на обучающей выборке. Для этого вводятся переменные $\xi_{i}$ характеризующих величину ошибки на объектах $x_i$ и коэффициент $C$ --- параметр, позволяющий настраивать отношение между максимизацией ширины разделяющей полосы и минимизацией суммарной ошибки. Тогда задача SVM формулируется как система (\ref{eq:RSVM3}). Формулы (\ref{eq:RSVM1})~--~ (\ref{eq:RSVM3})~\cite{ML_no_wors}.

Пусть имеется функция $f(x)$, которая определяет релевантность документа относительно запроса. Тогда утверждение о том, что документ $x_{i}$ релевантнее чем документ $x_{j}$  ($x_{i} \succ x_{j}$)  эквивалентно тому, что $f(x_{i}) > f(x_{j})$. Тогда если выбрать функцию $f(x)=(\omega, x)$ то имеем, что:
\[
	f(x_{i}) > f(x_{j}) \iff (\omega, x_{i} - x_{j}) > 0.
\]

Теперь рассматривая разность векторов как новые объекты $\hat{x}_{i, j} = x_{i} - x_{j}$, получаем стандартную постановку SVM алгоритма. Теперь задача ставиться следующим образом:
\begin{equation}
\begin{cases}
	\label{eq:RSVM4}
	\frac{1}{2}\| \omega \|^2+C  \displaystyle\sum_{i=1}^{N} \xi_{i}\to min_{\omega, \xi} \\
	y_{i}(\omega, x_{i}^1 - x_{i}^2) \leq 1 - \xi_{i} \\
	\xi_{i} \geq 0 \\
\end{cases},
\end{equation}
где $N$ --- количество пар объектов, $x_{i}^1$, $x_{i}^2$ --- первый и второй объект пары соответственно. При этом $y_{i} = 1$, если $x_{i}^1 \succ x_{i}^2$, иначе  $y_{i} = -1$. Допустимы парами являются документы из одного списка и с различной степенью релевантности. Формула (\ref{eq:RSVM4})~\cite{RankSVM}.

\subsection{LambdaRank}

В поточечных и попарных методах ранжирования итоговый функционал при обучении обычно не дифференцируемый, так как зависит от порядка элементов. В рассматриваемом алгоритме, описанном Бургес~\cite{LamdaRank}, вместо исходного функционала рассматривается непрерывный аналог, который легко оптимизировать.
Рассмотрим работу метода для одного поискового запроса. Алгоритм LambdaRank не определяет непрерывный приближенный функционал $L$, вместо этого он определяет градиент функционала на всем списке документов:
\begin{equation}
	\label{eq:LBR1}
	\frac{\partial L}{\partial s_{i}} = -\lambda(s_{1}, y_{1}, \dots, s_{n}, y_{n}),
\end{equation}
гдe $s_{i}$ --- рейтинг документа, $y_{i}$ --- степень релевантности, а $n$ --- количество документов. Градиент выбранного документу зависит от рейтингов и степени релевантности других\ и пересчитывается после каждой генерации списка.

Один из способов добиться повышения эффективности --- увеличение градиента документов на первых позициях, то есть более значимыми сделать перестановки с первыми элементами. Для двух документов $d_{i}$ и $d_{j}$ имеем, если $d_{i} \succ d_{j}$, то $| \frac{\partial L}{\partial s_{i}} | > | \frac{\partial L}{\partial s_{j}} |$.

Метод позволяет настраиваться на широкий класс функционалов качеств, однако в стандартном варианте $\lambda$ вычисляется по метрике NDCG:
\begin{equation}
	\label{eq:LBR2}
	\lambda_{i} = \frac{\partial L}{\partial s_{i}} = \frac{1}{G_{max}} \sum_{j}(\frac{1}{1 + exp(s_{j} - s{i})})(G(y_{j}) - G(y_{i}))(D(\pi_{j}) - D(\pi_{i})),
\end{equation}
где функции $G$, $D$ ---  функции преобразования релевантности документа в его рейтинг и порядковой номера документа в ранжированном списке.  $\lambda_{i}$ – показывает насколько надо увеличить рейтинг $i$-го документа. Для этого надо изменить веса $\omega$:
\begin{equation}
	\label{eq:LBR3}
	\frac{\partial L}{\partial \omega} = \sum_{i=1}^{n}\frac{\partial s_{i}}{\partial \omega}\sum_{j \in P_{i}}\frac{\partial L(s_{i}, s_{j})}{\partial s_{i}} + \sum_{j=1}^{n}\frac{\partial s_{j}}{\partial \omega}\sum_{i \in P_{j}}\frac{\partial L(s_{i}, s_{j})}{\partial s_{j}},
\end{equation}
где $P_{i}$, $P_{i}$ --- множество пар документов с индексами $j$ и $i$, для которых пары ($i$, $j$) во множестве пар документов $P$ соответственно.

Таким образом, алгоритм LambdaRank заключается в итерационном пересчете весов $\omega$:
\begin{equation}
	\label{eq:LBR4}
	\omega = \omega - \eta\frac{\partial L}{\partial \omega},
\end{equation}
где $\eta$ --- итерационный шаг. Если запросов несколько, то необходимо расширить множество $P$, которое может быть построено как и в методе Ranking SVM. Формулы(\ref{eq:LBR1})~--~(\ref{eq:LBR4})~\cite{LamdaRank}.

В данном методе для минимизации целевой функции используется градиентный спуск, суть которого заключается в использовании градиента для поиска локального минимума целевой функции. За счёт частных производных модели домноженных на -1 при каждой итерации её параметры пересчитываются таким образом, что происходит движение в сторону локального минимума~\cite{ML_no_wors}.

\subsection{ListNet}
Впервые алгоритм был описан З. Као~\cite{ListNet}.Цель обучения формализована как минимизация общих потерь в отношении обучающих данных.
\begin{equation}
	\label{eq:LN1}
	\sum_{i=1}^m L(y^{(i)}, z^{(i)}),
\end{equation}
где $L$ --- функция потерь по списку.
В данном алгоритме предлагается использовать вероятностные модели для вычисления функции потерь $L$ по списку.
Пусть набор объектов, подлежащих ранжированию, идентифицируется числами $1, 2, \dots, n$. Перестановка $\pi$ на объектах определяется как биекция от $\{1, 2, \dots, n\}$ к самой себе, т.~е. $\pi=(\pi(1), \dots, \pi(n))$. Здесь $\pi(j)$ обозначает объект в позиции $j$ в перестановке. Множество всех возможных перестановок $n$ объектов обозначается как $\Omega_n$. Пусть существует функция ранжирования, которая присваивает
оценки $n$ объектам. Обозначим список оценок как $s = (s_1, s_2, \dots, s_n)$, где $s_j$ --- оценка $j$-го объекта. 

Предполагается, что возможна любая перестановка, но разные перестановки могут иметь разную вероятность, рассчитанную на основе функции ранжирования. Определим вероятность перестановки таким образом, чтобы она обладала желаемыми свойствами. Тогда вероятность перестановки $\pi$, заданная списком оценок $s$, определяется как:
\begin{equation}
	\label{eq:LN2}
	P_s(\pi)=\prod_{j=1}^n \frac{\phi(s_{\pi(j)})}{\sum_{k=j}^n \phi(s_{\pi(k)})},
\end{equation}
где $\phi(s_{\pi(j)})$ --- это оценка объекта в позиции $j$ перестановки $\pi$, $\phi$ --- возрастающая и строго положительная функция.

Для любого списка ранжирования, основанного на данной функции, если поменять позицию объекта с высоким и низким баллом местами, получим список с меньшей вероятностью перестановки. Список объектов, отсортированных на основе функции ранжирования, имеет наибольшую вероятность перестановки, в то время как список объектов, отсортированных в обратном порядке, имеет наименьшую вероятность. То есть наиболее вероятной перестановкой является, отсортированная с помощью функции ранжирования.

Имея два списка оценок, мы можем сначала вычислить из них два распределения вероятностей перестановок, а затем вычислить расстояние между двумя распределениями как функцию потерь по списку. Однако, поскольку число перестановок равно $n!$, вычисление может оказаться крайне трудозатратным.

Чтобы справиться с данной проблемой, рассмотрим использование вероятности того, что $j$-ый документ будет ранжирован выше всех, учитывая оценки остальных документов
\begin{equation}
	\label{eq:LN3}
	P_s(j)=\frac{\phi(s_j)}{\sum_{k=1}^n \phi(s_k)} .
\end{equation}
Используя перекрестную энтропию в качестве метрики, функцию потерь~(\ref{eq:LN1}) можно записать в как:
\begin{equation}
	\label{eq:LN4}
	L(y^{(i)}, z^{(i)})=-\sum_{j=1}^n P_{y^{(i)}}(j) \log (P_{z^{(i)}}(j)).
\end{equation}

Обозначим функцию ранжирования, основанную на модели нейронной сети $\omega$ как $f_\omega$. Функция $f_\omega$ присваивает вектору признаков $x_j^{(i)}$ значение рейтинга. Определим $\phi$ как экспоненциальную функцию, тогда функцию вероятности~(\ref{eq:LN3}) можно записать как:
\begin{equation}
	\label{eq:LN5}
	P_s(j)=\frac{\phi(s_j)}{\sum_{k=1}^n \phi(s_k)}=\frac{\exp (s_j)}{\sum_{k=1}^n \exp (s_k)}.
\end{equation}

Имея запрос $q^{(i)}$ ранжирующая функция создаёт список рейтингов \[z^{(i)}(f_\omega)=(f_\omega(x_1^{(i)}), f_\omega(x_2^{(i)}), \cdots, f_\omega(x_{n^{(i)}}^{(i)})).
\]
Тогда функцию вероятности~(\ref{eq:LN5}) можно записать как:
\begin{equation}
	\label{eq:LN6}
	P_{z^{(i)}(f_\omega)}(x_j^{(i)})=\frac{\exp (f_\omega(x_j^{(i)}))}{\sum_{k=1}^{n^{(i)}} \exp (f_\omega(x_k^{(i)}))}.
\end{equation}

С перекрестной энтропией в качестве метрики, функции потери~(\ref{eq:LN4}) примет вид:
\begin{equation}
	\label{eq:LN7}
	L(y^{(i)}, z^{(i)}(f_\omega))=-\sum_{j=1}^{n^{(i)}} P_{y^{(i)}}(x_j^{(i)}) \log (P_{z^{(i)}(f_\omega)}(x_j^{(i)}))
\end{equation}

Градиент функции потери можно найти по следующей формуле:

\begin{equation}
	\label{eq:LN8}
	\begin{aligned}
		\Delta \omega= 
		\frac{\partial L(y^{(i)}, z^{(i)}(f_\omega))}{\partial \omega}=-\sum_{j=1}^{n^{(i)}} P_{y^{(i)}}(x_j^{(i)}) \frac{\partial f_\omega(x_j^{(i)})}{\partial \omega} \\
		+\frac{1}{\sum_{j=1}^{n^{(i)}} \exp (f_\omega(x_j^{(i)}))} \sum_{j=1}^{n^{(i)}} \exp (f_\omega(x_j^{(i)})) \frac{\partial f_\omega(x_j^{(i)})}{\partial \omega}
	\end{aligned}
\end{equation}


Метод ListNet заключается в итерационном пересчете $\Delta \omega$ и обновлении весов модели: $\omega = \omega - \eta\Delta \omega$. Используется метод градиентного спуска, описанный выше. Формулы~(\ref{eq:LN1})--(\ref{eq:LN8})~\cite{ListNet}