\chapter*{ВВЕДЕНИЕ}
\addcontentsline{toc}{chapter}{ВВЕДЕНИЕ}

Машинное обучение --- термин, обозначающий набор методов и инструментов, которые помогают компьютерам самостоятельно обучаться и адаптироваться. Алгоритмы машинного обучения помогают ИИ учиться, не будучи явно запрограммированным на выполнение желаемого действия, обрабатывая специально подобранные входные данные. Технологии связанные с машинным обучение стремительно развиваются и используются людьми во многих сферах жизни, например:

\begin{itemize}[label=---]
	\item распознавание изображений и речи;
	\item прогнозирование автомобильного трафика;
	\item рекомендательные системы;
	\item автопилот;
	\item медицина;
	\item виртуальные помощники;
	\item поисковые системы.
\end{itemize}

Актуальность этой темы объясняется широким распространением поисковых систем. Большинство людей ежедневно совершают множество запросов в поисках нужной информации, Google обрабатывает 3.5 миллиарда запросов в день или почти 40 тысяч запросов каждую секунду~\cite{google}. Для ускорения и улучшения качества поисковой выдачи применяются методы машинно обучения, а именно обучение ранжированию. Современные поисковые системы используют машинное обучение, например, Матрикснет в Яндексе.

Целью данной научно-исследовательской работы --- изучить алгоритмы машинного обучения, применяемые в поисковых системах. 

Для достижения поставленной в работе цели предстоит решить следующие задачи:
\begin{itemize}[label=---]
	\item изучить основные понятия алгоритмов обучения ранжированию;
	\item описать и классифицировать существующие алгоритмы;
	\item произвести сравнительный анализ рассмотренных алгоритмов.
\end{itemize}


