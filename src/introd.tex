\chapter*{ВВЕДЕНИЕ}
\addcontentsline{toc}{chapter}{ВВЕДЕНИЕ}

Машинное обучение --- термин, обозначающий набор методов и инструментов, которые помогают компьютерам самостоятельно обучаться и адаптироваться. Алгоритмы машинного обучения помогают ИИ учиться, не будучи явно запрограммированным на выполнение желаемого действия, обрабатывая специально подобранные входные данные. Технологии связанные с машинным обучение стремительно развиваются и используются людьми во многих сферах жизни, например:

\begin{itemize}[label=---]
	\item распознавание изображений и речи;
	\item прогнозирование автомобильного трафика;
	\item рекомендательные системы;
	\item автопилот;
	\item медицина;
	\item виртуальные помощники;
	\item поисковые системы;
\end{itemize}

Целью же данной научно--исследовательской работы является рассмотрение алгоритмов машинного обучения, в частности, их применение в поисковых системах.

Актуальность этой темы объясняется широким распространением поисковых систем. Большинство людей ежедневно совершают множество запросов в поисках нужной информации. Для ускорения и улучшения качества поисковой выдачи применяются методы машинно обучения, а именно обучение ранжированию. Современые поисковики используют машинное обучение, например, Матрикснет в Яндексе. 

Для достижения поставленной в работе цели предстоит решить следующие задачи:
\begin{itemize}[label=---]
    \item изучить основные понятия алгоритмов обучения ранжированию;
    \item ознакомиться с существующими алгоритмами и классифицировать их;
    \item произвести сравнительный анализ рассмотренных алгоритмов.
\end{itemize}


