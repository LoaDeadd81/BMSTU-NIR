\chapter{Анализ предметной области}

\section{Основные понятия}
Машинное обучение --- раздел информатики, посвященный созданию алгоритмов, опирающихся на набор данных о каком-либо явлении. Данные получают из естественной среды, создают вручную или сгенерируют другим алгоритмом. Методы машинного обучения формируют статистическую модель на основе специально подобранных обучающих данных, которую потом используются для решения практических задач~\cite{ML_no_wors}.

Выделяется три основных способа обучения: с учителем, без учителя и с подкреплением. Способы и их применение проиллюстрированы на рисунке~\ref{img:ml-learn}.

\imgScale{0.45}{ml-learn}{Способы обучения}

В обучении с учителем набор данных организован как коллекция размеченных образцов вида <<известный вход --- известный выход>>, называемый обучающей выборкой. Существует некоторая зависимость между входом и выходом, но она неизвестна. При обучении создаётся модель, отражающая эту зависимость, способная для любого входа выдать достаточно точный выход. Каждый вход является вектором признаков, описывающих некоторые характеристики образца. Выход может быть элементом конечного множества, вещественным числом или более сложной структурой. В основном используется для задач регрессии и классификации.

Обучение без учителя --- класс задач машинного обучения, в которых известны только описания множества объектов и требуется обнаружить внутренние взаимосвязи. На вход подаётся вектор признаков, а целью алгоритма обучения без учителя является --- создать модель, которая принимает входные параметры и преобразует их в значение, которое можно использовать для решения практической задачи. Используются для задач кластеризации, уменьшения размерности и выявления аномалий.

Обучение с подкреплением --- это раздел машинного обучения, в котором предполагается, что машина «живет» в определенном окружении и способна воспринимать его как вектор характеристик. Идея заключается в том, что компьютер взаимодействует со средой, параллельно обучаясь, и получает вознаграждение за выполнение действий т.~е. обучается методом проб и ошибок. Цель алгоритма обучения с подкреплением — выучить линию поведения, которая приводит к максимальному ожидаемому вознаграждению.


\section{Применение машинного обучения в поисковых системах}

Работа поисковой системы заключается в том, чтобы по запросу пользователя найти документы, наилучшим образом удовлетворяющие информационную потребность пользователя. При этом основными критериями качества результатов информационного поиска являются полнота, точность и оперативность поиска. Сайты в поисковой выдаче сортируются в соответствии с факторами, используемыми поисковым движком т.~е. проходят процедуру ранжирования.

Существует множество методов подбора формулы для ранжирования, но однин из самых популярных -- на основе машинного обучения, а именно  обучение ранжированию с учителем. Целью этих методов является подбор ранжирующей модели по обучающей выборке, которая способна наилучшим образом приблизить и обобщить способ ранжирования на новые данные. Для получения набора примеров используют асессоров, которые оцениваю степень релевантности документа запросу. Данные для обучения характеризуются большим объёмом (100 -- 500 тысяч примеров) и большой размерностью (100 -- 1000 признаков)~\cite{ML_for_rank}.


