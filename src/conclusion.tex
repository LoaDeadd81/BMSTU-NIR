\chapter*{ЗАКЛЮЧЕНИЕ}
\addcontentsline{toc}{chapter}{ЗАКЛЮЧЕНИЕ}

Цель, которая была поставлена в начале научно-исследовательской работы, была достигнута: рассмотрены алгоритмы машинного обучения, применяемые в поисковых системах.

Решены все поставленные задачи:
\begin{itemize}[label=---]
	\item изучены основные понятия алгоритмов обучения ранжированию;
	\item приведены существующие алгоритмы и классифицированы;
	\item произведён сравнительный анализ рассмотренных алгоритмов.
\end{itemize}


В ходе исследования были определены особенности, преимущества и недостатки рассмотренных методов обучения ранжированию.

Лучшим с точки зрения показателей метрик оказался алгоритм ListNet (списочный подход). В основном списочные алгоритмы превосходят остальные подходы по показателям метрик, но они сложны в разработке и подборе обучающих наборов данных.

Для алгоритмов с попарным подходом могут быть применены существующие методологии классификации непосредственно, что ускоряет и упрощает процесс их разработки. Однако у данного подхода есть ряд недостатков:
\begin{itemize}[label=---]
	\item целью обучения таких алгоритмов является минимизация ошибок в классификации пар документов, а не минимизация ошибок в ранжировании документов;
	\item процесс обучения требует больших вычислительных затрат, поскольку количество пар документов очень велико;
	\item количество сгенерированных пар документов в значительной степени варьируется от запроса к запросу, что приводит к обучению модели, ориентированной на запросы с большим количеством пар.
\end{itemize}

Алгоритмы же с поточенным подходом являются самыми простыми и наименее производительными. 